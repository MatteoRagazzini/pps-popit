\section{Sprint 1}
\subsection{Sprint Planning - 13/09/2021}
Il team si è incontrato via Teams per rivedere le specifiche e definire il primo sprint organizzativo.

\subsubsection{Obiettivi}
Gli obiettivi impostati per il primo sprint sono:
\begin{itemize}
    \item Realizzazione di una partita contenente tutte le entità di gioco, eventualmente rappresentate da delle forme geometriche; nonostante non abbiano una controparte visivamente identificabile, le entità riescono ad interagire tra di loro.
    \item Definizione di un modello MVC a scambio di messaggi.
    \item Creare un game loop basato su paradigma ad attori in grado di essere velocizzato, rallentato e stoppato e ripreso.
    \item Definizione di una mappa sulla quale giocare la partita.
    \item Implementazione di ogni entità (torri, palloncini e proiettili) e del corrispondente attore.
\end{itemize}
In questa fase sarà inoltre necessario riprendere consapevolezza degli strumenti di continuous integration visti a
lezione per poter seguire un flusso Git adeguato. Alla fine di questo sprint organizzativo si avrà quindi pronto lo
scheletro del progetto, il repository Git (già avviato per la continuous integration) e saranno stati decisi i
dettagli del processo di sviluppo.

\subsubsection{Suddivisione del lavoro}
In linea di massima, i task sono stati assegnati come segue:
\begin{itemize}
    \item Alessandro - GameLoop, Balloons, Balloon Types, Balloon Actor
    \item Simone - Model Actor, View Actor, Controller Actor, Maps
    \item Matteo - Entities, Bullet, Bullet Actor
    \item Tommaso - Towers, Tower Actor
\end{itemize}

\subsubsection{Continuous Integration}
Il team ha creato un repository privato su cui eseguire delle GitHub Actions includendo test banali, ma utilizzando le
principali librerie pensate in fase di progettazione per l'architettura del sistema. In particolare, sono stati
inclusi ScalaFX e Akka. Il lavoro è stato utile al team per accordarsi su come procedere durante lo sviluppo del
sistema e ha fornito un mezzo al committente per controllare l'usabilità del prodotto: per quanto riguarda la qualità
del codice, non è stato ancora inserito un controllo sui warning nè un linter all'interno del workflow, tuttavia sarà
uno degli obiettivi immediati dei successivi Daily Scrums.

\subsubsection{Repository GitHub}
Il team ha creato il repository del sistema e ha cominciato a lavorare sullo scheletro architetturale dell'applicativo,
in modo da poter iniziare a lavorare individualmente in maniera autonoma.

\subsection{Sprint Review - 26/09/2021}
Il team si è incontrato via Teams per commentare l'andamento dello sprint e discutere sulle modalità di lavoro adottate.

\subsubsection{Risultati}
Gli obiettivi impostati durante il primo sprint planning sono stati raggiunti; si evidenziano alcune particolarità:
\begin{itemize}
    \item le tracce di gioco sono state implementate con un generatore prolog casuale;
    \item è stata trovata una gerarchia di interfacce e classi che accomuna tutte le entità (uso intelligente dei mixin);
    \item è stato possibile testare il comportamento degli attori grazie a \texttt{ScalaTestWithActorTestKit};
    \item l'implementazione di Balloon e Tower richiederà dei cambiamenti per la complessità di queste ultime entità.
\end{itemize}

\subsubsection{Commenti}
Il gruppo è riuscito a collaborare senza problemi nonostante la distanza fisica e l'impossibilità di incontrarsi
quotidianamente ad orari fissi. Il lavoro dei singolo è stato integrato facilmente e con l'attenzione di tutti durante
i meeting organizzati. Si è notata una disparità del lavoro svolto, tuttavia la maggior parte del codice prodotto è
stato sviluppato in modalità pair programming, quindi tutti i membri del gruppo si ritengono soddisfatti per quanto
riguarda la suddivisione dei task. Il processo di sviluppo adottato sembra essere efficiente, ma ancora migliorabile.

\newpage