\section{Sprint 3}
\subsection{Sprint Planning - 13/10/2021}
Il team si è incontrato via Teams per definire il terzo sprint.

\subsubsection{Obiettivi}
Gli obiettivi impostati per il terzo sprint sono:
\begin{itemize}
    \item Avere una partita giocabile dall'inizio alla fine.
    \item Versione semplificata di un menù che dà la possibilità di navigare tra diverse pagine.
    \item Dare la possibilità di scegliere una mappa piuttosto che accettare quella data dal motore.
    \item Salvare nel file system le mappe giocate per poterle ri-giocare.
    \item Rifattorizzare il model, introducendo dei manager di messaggi per ogni logica interna al modello.
    \item Inserire animazioni per le esplosioni.
    \item Definire parametri di gioco come costo delle torri o dei power up.
    \item Implementazione della pagina delle tracce salvate e delle impostazioni di gioco.
\end{itemize}

\subsubsection{Suddivisione del lavoro}
In linea di massima, i task sono stati assegnati come segue:
\begin{itemize}
    \item Alessandro - Refactor Model, Spawner Manager, Entities Manager, Difficulty Levels
    \item Simone - Main Menu, Settings Page, Plot Different Tracks, Game Dynamics Manager, Balloon Patterns Blending
    \item Matteo - Animations, Saved Tracks Page, Freeze Balloons
    \item Tommaso - Track Serialization, Track Deserialization, Fix Game Parameters
\end{itemize}

\subsubsection{Report}
Un ulteriore obiettivo per questo sprint è aumentare la contribuzione alla relazione di progetto: l'idea è quella di
dover scrivere solamente le retrospettive nel quarto ed ultimo sprint; tuttavia, la quantità di obiettivi dello sprint
in procinto di iniziare è alta, quindi si considera tale obiettivo come opzionale.

\subsection{Sprint Revision - 29/10/2021}
Il team si è incontrato via Teams per commentare l'andamento dello sprint e discutere sulle modalità di lavoro adottate.

\subsubsection{Risultati}
Gli obiettivi impostati durante il secondo sprint planning sono stati raggiunti; si evidenziano alcune particolarità:
\begin{itemize}
    \item il refactor del model è stato talmente efficace che adesso esegue un semplice ridirezionamento dei messaggi ai manager interessati;
    \item per migliorare l'usabilità della pagina delle tracce salvate è stato introdotto uno screen shot delle tracce salvate;
    \item il salvataggio delle tracce viene fatto in dei file json all'interno di una cartella ".popit" nella home dell'utente;
    \item le parti del modello integrate con le difficoltà di gioco sono il plotter prolog delle tracce, lo spawner e la quantità di soldi guadagnati dallo scoppio dei palloncini;
    \item è stata studiata una sequenza incrementale di combinazioni di palloncini fino al round 50 delle partite;
    \item il gioco è stato reso "infinito", con combinazioni incrementali di tutti i tipi di palloncini oltre al round 50;
\end{itemize}

\subsubsection{Commenti}
Per via degli impegni personali e per la specificità dei task di ciascun membro del team, è stato più difficile del
solito coordinarsi; tuttavia, il gruppo è riuscito a mantenere una determinata frequenza per gli stand-up meeting e a
integrare le proprie parti del codice con quelle degli altri. Il pair programming è rimasta la modalità di lavoro
preferita del gruppo. Ci si ritiene molto soddisfatti dei raggiungimenti di questo terzo sprint.

\newpage