\section{Sprint 4}
\subsection{Sprint Planning - 29/10/2021}
Il team si è incontrato via Teams per definire il quarto sprint.

\subsubsection{Obiettivi}
Gli obiettivi impostati per il quarto sprint sono:
\begin{itemize}
    \item Definizione di uno stile per le pagine del menù, impostazioni e tracce salvate.
    \item Aggiunta di difficoltà, time ratio e frame rate alla pagina delle impostazioni.
    \item Aggiunta (opzionale) di un pulsante per l'inizio automatico del round.
    \item Refactor finale del codice (completare documentazione mancante, aumentare coverage, rimozione di eventuali bug).
    \item Completamento della relazione con diagrammi (fatti con plant uml) e con le parti ancora non scritte.
\end{itemize}

\subsubsection{Suddivisione del lavoro}
In linea di massima, i task sono stati assegnati come segue:
\begin{itemize}
    \item Alessandro - Time Settings, Refactor, Report
    \item Simone - Style, Refactor, Report
    \item Matteo - Automatic Round, Refactor, Report
    \item Tommaso - Power Up Limit, Refactor, Report
\end{itemize}

\subsection{Sprint Revision - 09/11/2021}
Il team si è incontrato via Teams per commentare l'andamento dello sprint e discutere sulle modalità di lavoro adottate.

\subsubsection{Risultati}
Gli obiettivi impostati durante il quarto sprint planning sono stati raggiunti; si evidenziano alcune particolarità:
\begin{itemize}
    \item si è cominciato a tracciare il backlog degli sprint e dei meeting;
    \item è stato cambiato il principale strumento per comporre diagrammi UML, adottando PlantUML;
    \item la feature opzionale dell'inizio automatico del round è stata inserita.
\end{itemize}

\subsubsection{Commenti}
Le tempistiche del progetto sono state pienamente rispettate, ed è stato effettuato l'ultimo meeting con qualche giorno
di anticipo: questo permetterà di avere una finestra di qualche giorno per revisionare in generale il sistema nella sua
interezza, insieme ai documenti allegati, prima dell'ultima release. Ogni membro del gruppo ha effettuato una
valutazione di se stesso e della cooperazione del team durante l'intero processo: complessivamente, si ritiene che tale
esperienza abbia formato il gruppo verso una metodologia agile con efficacia; il legame del team, già testato in lavori
precedenti, è stato consolidato ed è stato compreso da tutti il gusto personale degli altri.