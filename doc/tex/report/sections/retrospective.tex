\section{Retrospettiva}
In questa sezione viene valutato l'andamento complessivo del processo di sviluppo adottato, evidenziando eventuali
problemi riscontrati e apportando commenti finali.

\subsection{Sprint 1}
\subsubsection*{Svolgimento}
Nel primo sprint ci si è concentrati sulla corretta impostazione del progetto, andando a definire i dettagli del
processo di sviluppo, un flusso di continuous integration e l'architettura complessiva del sistema; inoltre, sono stati
definiti gli aspetti principali di design del modello, come la gerarchia di entità. Vista la potenziale espansione del
dominio trattato, in questo sprint ci si è confrontati continuamente sulle policy di gioco implementate in sprint
successivi per allineare la concezione comune del gioco dei diversi membri del team: questo ha anche portato a definire
una frequenza costante di meeting.

\subsubsection*{Commenti finali}
Data la natura dello sprint, incentrata più sul design che sull'implementazione, non sono stati riscontrati particolari
problemi durante lo svolgimento dello stesso: le tempistiche sono state rispettate, così come gli obiettivi e la
suddivisione del lavoro.

\subsection{Sprint 2}
\subsubsection*{Svolgimento}
Nel secondo sprint ci si è concentrati sul design di dettaglio e sull'implementazione di aspetti perimetrali, ma
essenziali in vista agli obiettivi impostati per lo stesso. La suddivisione del lavoro è stata più netta e separata
rispetto allo sprint precedente, ma è stata comunque trovata un'intersezione importante dei task. Inoltre, è stato
definito un flusso di continuous delivery e si è cominciato a compilare un product backlog relativo alla divisione dei
task tra i membri del team e al relativo effort.

\subsubsection*{Commenti finali}
Durante lo sprint è stato riscontrato un leggero rallentamento dovuto alla necessità di imparare ad utilizzare la
libreria ScalaFX; tuttavia, le tempistiche sono state rispettate, così come gli obiettivi e la suddivisione del lavoro.

\subsection{Sprint 3}
\subsubsection*{Svolgimento}
Nel terzo sprint ci si è posti come obiettivo la terminazione del gioco, per lasciare al quarto e ultimo sprint gli
aspetti meno essenziali. Nonostante ciò, è stato comunque possibile mantenere un approccio agile, incorporando comunque
cambiamenti anche a livello architetturale.
\subsubsection*{Commenti finali}
Un problema riscontrato durante lo svolgimento del terzo sprint è stata la difficoltà a coordinarsi, in particolare, è
stato più difficile organizzare i meeting rispetto agli sprint precedenti per via degli impegni personali di ciascun
membro; tuttavia, il gruppo è riuscito a cooperare grazie ad una costante comunicazione ed attività di "backlogging"
tramite l'utilizzo di strumenti come il product backlog e Trello. Nonostante la mole di task assegnati durante lo
sprint, le tempistiche sono state rispettate, così come gli obiettivi e la suddivisione del lavoro.

\subsection{Sprint 4}
\subsubsection*{Svolgimento}
Nel quarto sprint ci si è concentrati sulla rifinitura di aspetti perimetrali come lo stile delle pagine o le
impostazioni di gioco, andando ad aggiungere feature non preventivate dall'analisi dei requisiti, ma che possono essere
ritenute implicite. Inoltre, si è pensato di utilizzare una parte di tempo di tale sprint per redigere le parti
restanti della relazione e per risolvere eventuale debito tecnico accumulato.

\subsubsection*{Commenti finali}
Le tempistiche sono state rispettate, così come gli obiettivi e la suddivisione del lavoro. Il gruppo si ritiene molto
soddisfatto del sistema finale ottenuto, ma, complessivamente, dell'intero processo di sviluppo che ha portato ai
risultati raggiunti. Ciascun membro del gruppo ha valutato la propria quantità di debito tecnico accumulato e, in media,
è stato raggiunto un livello soddisfacente: gli unici aspetti da completare sono stati un po' di documentazione e di
coverage.
