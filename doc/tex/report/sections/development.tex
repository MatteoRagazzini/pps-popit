\section{Processo di sviluppo}
\label{sec:development}
Il processo di sviluppo adottato dal team è fortemente ispirato a \textbf{Scrum},
basato su \textbf{sprint} e \textbf{obiettivi}, per realizzare un progetto in maniera \textbf{agile},
ma tenendo conto della presenza di ruoli come il \textit{committente} e il \textit{product owner} tra gli sviluppatori.
Il team effettua sprint della durata di circa due settimane, durante i quali vengono inizialmente decisi gli obiettivi,
suddivisi i compiti e svolti incontri frequenti per discutere del lavoro portato a termine.
Di seguito, si discutono gli elementi fondamentali del processo di sviluppo adottato.

\subsection{Meeting}
I meeting sono un fattore fondamentale per il processo di sviluppo e avvengono con cadenza quasi giornaliera e con durate differenti in base all'importanza dello stesso. Viste le difficoltà apportate dalla pandemia, la maggior parte degli incontri è avvenuto a distanza, utilizzando \textit{Microsoft Teams} come piattaforma principale.

\subsubsection{Meeting iniziale}
Prima della proposta del progetto, è avvenuto un incontro all'interno del quale sono stati decisi i seguenti fattori essenziali:
\begin{itemize}
    \item \textbf{Ruoli interni al team} - colui che ha proposto il progetto è stato eletto committente, mentre è stato scelto a preferenza il \textit{product owner}.
    \item \textbf{Specifiche del progetto} - sono stati decisi gli obbiettivi funzionali del progetto, facendo particolare attenzione alla loro fattibilità e soppesando le abilità del gruppo già testate in precedenza; le specifiche sono state trascritte in un documento formale redatto dal \textit{product owner} trasformando il linguaggio naturale del committente in obiettivi dettagliati.
    \item \textbf{Primo sprint} - è stato decisa l'organizzazione del primo sprint, definendo l'obiettivo finale e suddividendo dei \textit{task} tra i componenti del gruppo.
\end{itemize}

La durata del primo meeting è stata di circa 3 ore.

\subsubsection{Sprint Planning}
All'inizio di ogni di sprint viene effettuato un incontro all'interno del quale si discutono i risultati dello sprint precedente e si definiscono gli obiettivi di quello successivo. Nello specifico, i fattori su cui si vuole porre particolare attenzione sono:
\begin{itemize}
    \item Definizione degli \textbf{obiettivi}.
    \item Definizione ed assegnazione dei \textbf{task}.
    \item Valutazione dell'\textbf{andamento complessivo} del progetto, rimarcando eventuali ritardi.
    \item Valutazione dello \textbf{sprint precedente}.
    \item Valutazione della modalità di sviluppo \textbf{agile}, suggerendo miglioramenti.
\end{itemize}
La durata ideale degli sprint planning è fissata a 2 ore.

\subsubsection{Stand-up Meeting}
Con cadenza quasi giornaliera, il team effettua degli incontri in cui ogni membro espone il lavoro portato a termine, dichiarando, senza andare nel dettaglio, eventuali difficoltà. La durata ideale degli stand-up meeting è fissata a 20 minuti.

\subsection{Tool Ausiliari}
A supporto del processo agile, il team si impone di utilizzare strumenti con lo scopo di migliorare l’efficienza e di consentire al gruppo di concentrarsi maggiormente sul processo di sviluppo.

\subsubsection{Automazione}
Sono stati adoperati i seguenti processi:
\begin{itemize}
    \item \textbf{Test Driven Development} - scrivere in anticipo dei test per il proprio codice, permette di intercettare eventuali errori che insorgono durante l'integrazione del lavoro reciproco.
    \item \textbf{Continuous Integration} - per verificare la compatibilità e la correttezza del software prodotto viene sfruttata la funzionalità \textbf{GitHub Actions}, definendo all'inizio del processo di sviluppo un preciso workflow che assicura la corretta manutenzione del progetto.
    \item \textbf{Continuous Delivery} - per evitare di produrre manualmente gli \textit{artifacts} alla fine di ogni Sprint, è stato individuato un workflow adatto che permettesse di creare delle Release automatiche ogni qual volta il branch principale di git (\textit{main}) venisse aggiornato.
\end{itemize}

\subsubsection{Coordinazione}
La comunicazione è fondamentale per un processo di sviluppo agile, anche se i membri del team si conoscono a fondo. Per coordinarsi al meglio, il team ha deciso di utilizzare \textbf{Trello}, con il quale vengono tracciati i task dei singoli membri con il rispettivo andamento, individuando un flusso di lavoro all'interno di ogni sprint organizzativo. Inoltre, il \textit{product owner} del team ha redatto un \textit{product backlog} nel quale si è tenuto traccia dei task portati a termine da ciascun membro del gruppo, indicando per ciascuno il costo in termini di tempo e difficoltà di progettazione e/o implementazione.
