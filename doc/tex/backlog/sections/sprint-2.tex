\section{Sprint 2}
\subsection{Sprint Planning - 26/09/2021}
Il team si è incontrato via Teams per definire il secondo sprint.

\subsubsection{Obiettivi}
Gli obiettivi impostati per il secondo sprint sono:
\begin{itemize}
    \item Definizione di un'interfaccia grafica completa e funzionante in cui tutte le entità vengono mostrate e riescono a interagire tra di loro.
    \item Implementazione di un menù laterale contentente le torri che possono essere selezionate e piazzate sulla mappa.
    \item Revisione dell'implementazione di torri e palloncini.
    \item Implementazione dei power up delle torri.
    \item Implementazione dello spawner tramite una DSL che faciliti la scrittura dello spawn dei palloncini.
    \item Gestione della collision detection tra proiettili e palloncini.
    \item Terminare l'implementazione di tutti i tipi di entità di gioco.
\end{itemize}

\subsubsection{Suddivisione del lavoro}
In linea di massima, i task sono stati assegnati come segue:
\begin{itemize}
    \item Alessandro - Balloon Types, Spawner DSL
    \item Simone - ScalaFX and fxml, Lateral Menu, Rendering DSL
    \item Matteo - Bullet Types, Bullet Actor Death Policy, Collision Detection
    \item Tommaso - Tower Revision, Tower Types, Tower Power Ups
\end{itemize}

\subsubsection{Continuous Delivery}
Durante questa fase sarà inoltre necessario definire un workflow di continuous delivery per riuscire a produrre un
jar eseguibile per ogni sistema operativo vista la dipendenza da ScalaFX; tale necessità è fuoriuscita durante la
scorsa sprint review, quando è stata fatta la prima release del sistema: si ritiene che tale tool di automazione
semplifichi la procedura di deployment e risparmi tempo al team.

\subsubsection{Product Backlog}
All'inizio del secondo sprint verrà compilato un product backlog da ogni membro del gruppo, nel quale si indicheranno
i task svolti e l'impegno impiegato durante ciascuno. Tale documento verrà aggiornato continuamente anche nel corso
dei successivi sprint.

\subsubsection{Coverage}
Il committente ha richiesto di aumentare la coverage del sistema con dei test, per cui sarà necessario
inserire nel progetto Scoverage ed utilizzarlo per testare al meglio il software prodotto.

\subsubsection{Linter}
Il committente ha richiesto che il codice prodotto venisse trattato con un linter a piacere, per cui sarà necessario
introdurre nel progetto scalafmt per formattare il codice secondo una configurazione standard e comune a tutti i membri
del gruppo.

\subsection{Sprint Revision - 13/10/2021}
Il team si è incontrato via Teams per commentare l'andamento dello sprint e discutere sulle modalità di lavoro adottate.

\subsubsection{Risultati}
Gli obiettivi impostati durante il secondo sprint planning sono stati raggiunti; si evidenziano alcune particolarità:
\begin{itemize}
    \item è stato necessario cambiare il metodo di disegnamento delle immagini a interfaccia in quanto l'eccessiva attività di input/output ha causato rallentamenti notevoli del gioco;
    \item il paradigma ad attori si sta rivelando estremamente potente nel sistema, tuttavia stanno aumentando a dismisura i messaggi all'interno del modello;
    \item le torri sono state implementate come generiche sul proiettile in quanto dipendono esclusivamente da esso;
    \item siccome ci si sta concentrando maggiormente sul modello del gioco, si sta pensando di terminare la logica delle partite durante il terzo sprint e utilizzare il quarto per definire elementi meno fondamentali come la navigazione tra menù o lo stile.
\end{itemize}

\subsubsection{Commenti}
Durante questo sprint è stato più facile coordinarsi rispetto a quello precedente, vista l'esperienza acquisita durante
lo stesso. La suddivisione del lavoro non è stata rispettata a pieno: alcuni task non sono stati interamente portati a
termine dallo sviluppatore designato, ma comunque il pair programming rimane la modalità di lavoro preferita dal gruppo:
per la maggior parte dei task, ci si è riuniti e si è ragionato su quale fosse il metodo migliore per portarli a
termine, discutendo aspetti di design e di implementazione insieme per poi passare alla stesura di codice e test. Uno
degli obiettivi impliciti dello sprint (per ciascuno) era provare a rispettare il più possibile il test driven
development: per alcuni è risultato più facile che per altri. Il gruppo è complessivamente soddisfatto dei risultati
raggiunti finora e delle modalità di sviluppo.

\newpage
